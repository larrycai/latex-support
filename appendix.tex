\chapter{培训示例模板}
\label{培训示例模板}

如果在公司培训,这是一个模板,建议根据实际情况修改。

\section{时间安排}
\label{时间安排}

所有的培训都在二楼的哥德堡培训室,请带好自己的电脑,准时到达。

\begin{verbatim}
第一天  09:00-09:05 培训介绍
        09:05-10:35 企业组织结构介绍、敏捷、Scrum
        10:35-11:30 布置练习,在白板前实战学习。
        13:00-16:00 Git入门,代码审阅与Gerrit。
第二天  09:00-09:15 站立会议,练习任务点评。
        09:20-11:20 持续集成(java,maven,jenkins)
        13:00-16:00 Java质量,Game of life介绍
第三天  09:00-11:20 实例化需求
        13:00-16:00 Cucumber
第四天  09:00-10:00 Game of life项目任务和计划会议
        10:00-15:00 团队任务
        15:00-16:00 任务审阅和回顾
        16:00-16:30 结束,颁发毕业证书,拍照留念 
\end{verbatim}

\section{培训的材料}
\label{培训的材料}

培训的机器是Windows平台

\begin{itemize}
\item 所有的培训Slides:\href{http://server/download/ppt}{http:/\slash server\slash download\slash ppt}

\item 所有的需要安装的软件在:\href{http://server/download/software}{http:/\slash server\slash download\slash software} (第一天用U盘拷贝到培训员工各自桌面)

\end{itemize}

这本手册只是一个参考,你需要花更多的时间去了解熟悉它们,4天的学习会很有意思,要有激情,否者什么都学不到。

\section{培训负责人准备工作}
\label{培训负责人准备工作}

\begin{itemize}
\item 培训房间,网络,老师安排,会议通知,手册打印。

\item Gerrit服务器和Jenkins服务器和tomcat服务器

\item \href{https://github.com/wakaleo/game-of-life}{生命游戏(game of life)}\footnote{\href{https://github.com/wakaleo/game-of-life}{https:/\slash github.com\slash wakaleo\slash game-of-life}}Git仓库复原

\end{itemize}

\section{反馈}
\label{反馈}

作为企业中的一员,学会给别人反馈是第一要务,给这些课提些建议吧。谢谢。

\chapter{企业版本控制的改革:走向Git}
\label{企业版本控制的改革:走向git}

在传统企业中,版本控制系统大都采用ClearCase或SVN。特别是ClearCase在早期提供了强大的企业应用的功能,我们部门也很早使用了。而且长久以来,在它周围建立了无数的应用和流程,同事们都觉得它是必须的了。

然而随着敏捷和开放的推动下,在有些产品用ClearCase开发碰到了很多局限,比如在家上班,远程团队开发。有人开始想到是否可以引入其他工具来解决,不过在大型企业中要改变这种基础的工具是很难的。

我就想介绍一下我们是如何一步步地走向Git的。

特别声明:本文原为图灵社区活动“\href{http://www.ituring.com.cn/activity/details/696}{唤醒你心中的布道师}\footnote{\href{http://www.ituring.com.cn/activity/details/696}{http:/\slash www.ituring.com.cn\slash activity\slash details\slash 696}}”而写的文章:\href{http://www.ituring.com.cn/article/details/721}{企业版本控制的改革:从ClearCase到Git--我的布道之旅}\footnote{\href{http://www.ituring.com.cn/article/details/721}{http:/\slash www.ituring.com.cn\slash article\slash details\slash 721}},这里有所编辑。

这里想说的主要是如何在需要的时候推动技术的变革,而不是探讨技术的好坏。每个技术都有适应的场所,请勿生搬硬套。

关于版本控制的选择,也可以看看Martin Fowler写的\href{http://martinfowler.com/bliki/VersionControlTools.html}{版本控制工具(english)}\footnote{\href{http://martinfowler.com/bliki/VersionControlTools.html}{http:/\slash martinfowler.com\slash bliki\slash VersionControlTools.html}}

\section{了解最新技术-分布式版本控制(DVCS)}
\label{了解最新技术-分布式版本控制(dvcs)}

在推动技术改变的时候,首先要了解最新的技术状况,别学了一个旧了过时了的。

我们使用的主要是ClearCase,开始考虑这个转换的时候是在2009年初,SVN是第一个考虑的对象,因为它在开源中用的最多,\href{http://sourceforge.net}{sourceforge}\footnote{\href{http://sourceforge.net}{http:/\slash sourceforge.net}}和Eclipse的很多项目多用它,但我总觉得缺了点什么。

恰好我有个同事提到SVN和ClearCase都是集中式的,推荐我看看一个分布式版本控制工具:Mercurial,说实话听了介绍不是很懂,没有眼前一亮的感觉。聊了一下,感觉和SVN的分支没有多大区别,何况DVCS还需要两层提交呢。

同时我也了解到还有其它的分布式版本控制工具Git,Bazaar可供选择。

不管怎么样,我了解到这块领域有了最新的技术,它或许能解决我们的问题(要不时地问问自己为什么)。

\section{尝试在日常中使用分布式版本控制}
\label{尝试在日常中使用分布式版本控制}

为了尽快了解DVCS,我决定要在日常的开发中用用它,实践它,尽快地掌握它的关键。

由于同事对Mercurial很熟,我就踏踏实实地用Mercurial尝试了两个星期,不懂就问他,顺便查查资料去比较一番。

DVCS真是很神奇,很好用,特别对我的胃口,感觉DVCS天生是为软件开发用的。

在同一时刻,我又比较深入的看了看其他的系统如Git,发现Git的生态圈更好一点。在软件开发中,生态圈会决定将来这个工具的发展趋势。

\begin{itemize}
\item 如Eclipse插件开发邮件中开始讨论并决定用Git替代svn。

\item Git有很多的书可供选择(如 \href{http://progit.org/}{ProGit}\footnote{\href{http://progit.org/}{http:/\slash progit.org\slash }}),\href{http://git-scm.com/}{git在线网站}\footnote{\href{http://git-scm.com/}{http:/\slash git-scm.com\slash }}的内容也极其丰富。

\item \href{https://github.com/}{github}\footnote{\href{https://github.com/}{https:/\slash github.com\slash }}也漂亮得提供git的支持。补充一下,那时候\href{http://bitbucket.org/}{bitbucket}\footnote{\href{http://bitbucket.org/}{http:/\slash bitbucket.org\slash }}和github还在同一个水平线上。\href{http://code.google.com/}{google code}\footnote{\href{http://code.google.com/}{http:/\slash code.google.com\slash }}也还不支持git,只有Mercurial和svn。

\end{itemize}

通过这些实践和了解,发现DVCS-Git很适合我们所在的部门的企业产品软件开发。

\section{宣扬和推广分布式版本控制}
\label{宣扬和推广分布式版本控制}

要在企业中换一个版本控制工具难度非常大,所以必须要布道,我采用了下面的方法:

\begin{enumerate}
\item 每月我们都有固定学习新东西的时间,我就推荐了Mercurial、Git两个课程,让大家共同来学习,了解它。顺便我要看看开发者对它的接受程度,有趣的是,水平越牛的人越是喜欢它,纷纷过来问什么时候能在产品开发中用上Git。

\item 除了开发者,管理者和其他的使用者(配置管理的同事)的想法也很重要。我经常抓住机会和这些人聊DVCS ,聊Git,给他们介绍,看看他们有什么想法。当然他们有时候会不同意我的观点(有强势的,有委婉的),我就试图去说服他们,并从中挖掘出推动这个变化的关键因素。

\end{enumerate}

慢慢得我就得到了很多如何推动这个变化的关键说服点,这个每次情况都不会一样。

\section{详细研究版本迁移}
\label{详细研究版本迁移}

开发者想使用分布式版本控制的呼声越来越高,管理者也开始认真考虑了。

在企业中,改变所需要的研究评测报告是必不可少的了,这也给了我一次重新认识集中式和分布式版本控制的过程,我花了更多的时间去想这个改变对企业带来的好处。实际上开发者有时候不会考虑到整个软件开发的所有方面,如安全,持续集成等等。报告的大致框架是:

\begin{itemize}
\item 现在问题是什么?

\item 什么是DVCS,Git是什么?

\item 能改变什么?带来的好处?

\item 如果变化,计划是什么?

\end{itemize}

这一期间,使我静下心来更详细地了解了Git对企业可能的影响(有好的,有坏的),并制定了相应的对策。

\section{开始在小范围实施}
\label{开始在小范围实施}

技术改变需要耐心和机遇,机缘巧合,迁移到Git的建议比较顺利地被管理层接受了。

然后就是要去认认真真地实施了,这不是一个小问题,既然是软件开发,来不得半点的马虎,细节决定一切。而且实施得好坏还涉及到产品开发的正常运转。

企业中一般会选择从小范围开始实施,成功了才推广,下面是我们的一些实践。

\begin{itemize}
\item 我们开始用\href{https://github.com/sitaramc/gitolite}{gitolite}\footnote{\href{https://github.com/sitaramc/gitolite}{https:/\slash github.com\slash sitaramc\slash gitolite}}作为Git服务器,架好试验平台,在一个小项目中开始尝试。

\item 人手一本Git的书,安排Git入门培训,提高驾驭Git的能力。

\item 不断收集资料,提高对Git的认识。

\end{itemize}

还好基本上没有出大的差错,虽然有蛮多技术难点的,不过最后都解决了。通过小范围的使用推广,我们的技术储备也加强了(特别是配置管理的人),对下一步的全面实施更有信心。

\section{推广、并引入Gerrit做代码审查}
\label{推广、并引入gerrit做代码审查}

早期我们用的是gitolite来架Git服务器,它很不错。不过后来发现\href{http://code.google.com/p/gerrit/}{Gerrit}\footnote{\href{http://code.google.com/p/gerrit/}{http:/\slash code.google.com\slash p\slash gerrit\slash }}更好用,后来就切换过去使用了。这一点很重要,要不断探索这些新技术,争取在大规模推广前,用一个最适合的工具,否则一用上,在企业中就很难改变了。

Git开始在更多团队和更多产品中使用后,我们不断加强知识的培训,而且把相关的系统(如持续集成)都迁移到Git上去。一切都还不错,只是Git比想象中还复杂一点。

因为Gerrit有很强大的代码审查(code review)功能,不久以后这个功能也用上去了,代码提交的质量一下子上了一个档次,这是开始推动Git变革时没有想到的。

\section{小结}
\label{小结}

技术的变化不是那么容易得,需要天时、地利、人和,缺一不可。如果你有什么好建议,欢迎一起探讨。
